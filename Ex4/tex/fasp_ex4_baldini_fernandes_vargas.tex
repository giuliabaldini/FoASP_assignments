\documentclass[12pt]{article}

%% Language and font encodings
\usepackage[english]{babel}
\usepackage[utf8x]{inputenc}
\usepackage[T1]{fontenc}
\usepackage{fancyhdr}

%% Sets page size and margins
\usepackage[a4paper,top=3cm,bottom=2cm,left=3cm,right=3cm,marginparwidth=1.75cm]{geometry}

%% Packages
\usepackage{xcolor}
\usepackage[colorlinks=true, allcolors=red]{hyperref}
\usepackage{lipsum}
\usepackage{graphicx}
\usepackage{float}
\usepackage[all]{hypcap}
\usepackage{changepage}
\usepackage{amsmath}
\usepackage{amsthm}
\usepackage{amssymb}
\usepackage{xspace}
\usepackage{tikz}
\usepackage{amsmath,amsfonts,amssymb,amsthm}
\usepackage{mathtools}

%% Other
\graphicspath{{Figures/}}
\setlength\parindent{0pt}
\newcommand{\auth}{Giulia Baldini, Luis Fernandes, Agustin Vargas Toro}
\newcommand{\ass}{Assignment 4}

%% Page settings
\pagestyle{fancy}
\fancyhf{}
\rhead{\auth}
\lhead{\ass}
\rfoot{\thepage}

\title{Foundations of Audio Signal Processing\\ \ass}
\author{\auth}

\begin{document}
	\maketitle
	\section*{Exercise 4.1}
	\begin{alignat*}{3}
	\Vert \sum_{j=1}^{n} x_j\Vert^2 &= \langle\sum_{j=1}^{n} x_j, \sum_{j=1}^{n} x_j\rangle & \text{ Scalar product norm definition}\\
	&=\sum_{j=1}^{n} \langle x_j, x_j\rangle & \text{ Linearity of scalar product}\\
	&=\sum_{j=1}^{n} \Vert x_j \Vert^2 & \text{ Scalar product norm definition}
	\end{alignat*}
	\section*{Exercise 4.2}
	\textbf{a.} $d(x,y) = \vert x - y \vert$\\
	Let $x = a_1 + b_1 i, y = a_2 + b_2 i, z = a_3 + b_3 i, \forall x, y, z \in \mathcal{C} $ then
	\begin{equation*}
		\vert x - y \vert = \sqrt{(a_1 + b_1)^2 - (a_2 + b_2)^2} = \sqrt{(a_1 - a_
			2)^2 + (b_1 - b_2)^2}
	\end{equation*}
	1. $\sqrt{(a_1 - a_2)^2 + (b_1 - b_2)^2} \geq 0$ holds, as the the square root of a positive number is always a positive number (resp. $\sqrt0 = 0$)\\
	2. $\sqrt{(a_1 - a_2)^2 + (b_1 - b_2)^2} = 0$ only if $a_1 = a_2$ and $b_1 = b_2$, which is the case when $x=y$.\\
	3. $\sqrt{(a_1 - a_2)^2 + (b_1 - b_2)^2} = \sqrt{(a_2 - a_1)^2 + (b_2 - b_1)^2}$\\
	4. $d(x,y) + d(y,z) = \sqrt{(a_1 - a_2)^2 + (b_1 - b_2)^2} + \sqrt{(a_2 - a_3)^2 + (b_2 - b_3)^2}$.\\
	According to Minkowsky Inequality:
	\begin{equation*}
	\left(\sum _{{k=1}}^{n}|x_{k}+y_{k}|^{p}\right)^{{{\frac  {1}{p}}}}\leq \left(\sum _{{k=1}}^{n}|x_{k}|^{p}\right)^{{{\frac  {1}{p}}}}+\left(\sum _{{k=1}}^{n}|y_{k}|^{p}\right)^{{{\frac  {1}{p}}}}
	\end{equation*}
	If we consider the case with $p=2$ and $n=2$ (in our case $x_1 = (a_1 - a_2)$ and $x_2 = (b_1 - b_2)$ and similarly are $y_1$ and $y_2$ defined), then:
	$$d(x,y) + d(y,z) \geq \sqrt{|x_1 + y_1|^2 + |x_2 + y_2|^2}  $$
	Since in our case $x_k,y_k \in \mathbb{R}$:
	$$d(x,y) + d(y,z) \geq \sqrt{(a_1 - a_2 + a_2 - a_3)^2 + (b_1 - b_2 + b_2 - b_3)^2}  $$
	$$d(x,y) + d(y,z) \geq \sqrt{(a_1 - a_3)^2 + (b_1 - b_3)^2}  $$
	$$d(x,y) + d(y,z)  \geq d(x,z)$$\\\\
	\textbf{b.} $d(x,y) = \vert x \vert \cdot \vert y \vert$\\
	In this case the second property ($d(x, y) = 0 \text{ iff } x = y$) does not hold because $d(x,y) = 0 \text{ with } x = 0 \text{ and } y = 2 - 5i$.\\\\
	\textbf{c.} $d(x,y) = \begin{cases} 1 & x \neq y\\ 0 & \mbox{else} \end{cases}$\\
	$\forall x, y, z \in \mathcal{C}:$\\
	1. $d(x,y) \geq 0$ holds, because the possible values are $0,1$.\\
	2. $d(x, y) = 0 \text{ iff } x = y$ holds for the definition of $d(x,y)$.\\
	3. $d(x, y) = d(y,x)$ holds, because $x \neq y$ and $y \neq x$ are the same.\\
	4. $d(x, z) \leq d(x, y) + d(y, z)$ holds because if $ x \neq y \neq z$ then $1 \leq 1 + 1$. If $x = y \neq z$ then $1 \leq 0 + 1$. If $x \neq y = z$ then $1 \leq 1 + 0$. If $x = z \neq y$ then $0 \leq 1 + 1$. If $x = y = z$ then $0 \leq 0 + 0$.
	
	\section*{Exercise 4.3}
	\textbf{a-b.} The solutions can be found inside the \texttt{code} folder.
	\section*{Exercise 4.4}
	\textbf{a-b.} The solutions can be found inside the \texttt{code} folder.
\end{document}
