\documentclass[12pt]{article}

%% Language and font encodings
\usepackage[english]{babel}
\usepackage[utf8x]{inputenc}
\usepackage[T1]{fontenc}
\usepackage{fancyhdr}

%% Sets page size and margins
\usepackage[a4paper,top=3cm,bottom=2cm,left=3cm,right=3cm,marginparwidth=1.75cm]{geometry}

%% Packages
\usepackage{xcolor}
\usepackage[colorlinks=true, allcolors=red]{hyperref}
\usepackage{lipsum}
\usepackage{graphicx}
\usepackage{float}
\usepackage[all]{hypcap}
\usepackage{changepage}
\usepackage{amsmath}
\usepackage{amssymb}
\usepackage{xspace}
\usepackage{tikz}

%% Other
\graphicspath{{Figures/}}
\setlength\parindent{0pt}
\newcommand{\auth}{Giulia Baldini, Luis Fernandes, Agustin Vargas}
\newcommand{\ass}{Assignment 3}

%% Page settings
\pagestyle{fancy}
\fancyhf{}
\rhead{\auth}
\lhead{\ass}
\rfoot{\thepage}

\title{Foundations of Audio Signal Processing\\ \ass}
\author{\auth}

\begin{document}
	\maketitle
	\section*{Exercise 3.1}
	\textbf{a.}
	\begin{alignat*}{2}
	4 + i 4\sqrt{3} &= 1 + i \sqrt{3}
	\end{alignat*}
	$a = 1$, $b = \sqrt{3}$
	\begin{alignat*}{2}
		r &= \sqrt{1 + 3} = 2\\
		\cos\phi &= \frac{1}{2}\\
		\sin\phi &= \frac{\sqrt{3}}{2}\\
		\phi &= \frac{\pi}{3}
	\end{alignat*}
	\textbf{b.}
	\begin{alignat*}{3}
		(-1 + i \sqrt{3})^4 &= (1 - i 2 \sqrt{3} - 3)^2\\
		&= (- 2 - i 2 \sqrt{3})^2\\
		&= 4 (1 + i 2 \sqrt{3} - 3)\\
		&= 4 (-2 + i 2 \sqrt{3}) &= -8 + i 8 \sqrt{3} 
	\end{alignat*}
	$a = -8$, $b = 8\sqrt{3}$
	\begin{alignat*}{2}
	r &= \sqrt{64 + 192} = 16\\
	\cos\phi &= -\frac{8}{16} = -\frac{1}{2}\\
	\sin\phi &= \frac{8\sqrt{3}}{16} = \frac{\sqrt{3}}{2}\\
	\phi &= \frac{2\pi}{3}
	\end{alignat*}
	\textbf{c.} Here we use the solution from exercise b to solve the numerator.
	\begin{alignat*}{3}
	\frac{(-1 + i \sqrt{3})^4}{4 + i 4 \sqrt{3}} &= \frac{-8 + i 8 \sqrt{3}}{4 + i 4 \sqrt{3}}\\
	&= \frac{-2 + i 2 \sqrt{3}}{1 + i \sqrt{3}}\\
	&= \frac{(-2 + i 2 \sqrt{3})(1 - i \sqrt{3})}{(1 + i \sqrt{3})(1 - i \sqrt{3})}\\
	&= \frac{-2 + i 2 \sqrt{3} + i 2\sqrt{3} + 6}{(1 + 3)}\\
	&= \frac{4 + i 4\sqrt{3}}{4} &= 1 + i \sqrt{3}
	\end{alignat*}
	$a = 1$, $b = \sqrt{3}$, which are the same as in exercise a, and thus lead to the same solution.\\
	\textbf{d.}
	\begin{alignat*}{3}
	2 e^{\frac{\pi}{2} i}(1 + i ) &= 2 (\cos\frac{\pi}{2} + i\sin \frac{\pi}{2})(1 + i)\\
	&= 2(0 + i)(1 + i) &= -2 + 2i
	\end{alignat*}
	$a = -2$, $b = 2$
	\begin{alignat*}{2}
	r &= \sqrt{4 + 4} = 2\sqrt{2}\\
	\cos\phi &= -\frac{2}{2\sqrt{2}} = -\frac{\sqrt{2}}{2}\\
	\sin\phi &= \frac{2}{2\sqrt{2}} = \frac{\sqrt{2}}{2}\\
	\phi &= \frac{3\pi}{4}
	\end{alignat*}
	\section*{Exercise 3.2}
	\textbf{a.}
	\section*{Exercise 3.3}
	\textbf{a.}

\end{document}
