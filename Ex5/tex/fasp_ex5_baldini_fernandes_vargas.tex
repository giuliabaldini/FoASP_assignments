\documentclass[12pt]{article}

%% Language and font encodings
\usepackage[english]{babel}
\usepackage[utf8x]{inputenc}
\usepackage[T1]{fontenc}
\usepackage{fancyhdr}

%% Sets page size and margins
\usepackage[a4paper,top=3cm,bottom=2cm,left=3cm,right=3cm,marginparwidth=1.75cm]{geometry}

%% Packages
\usepackage{xcolor}
\usepackage[colorlinks=true, allcolors=red]{hyperref}
\usepackage{lipsum}
\usepackage{graphicx}
\usepackage{float}
\usepackage[all]{hypcap}
\usepackage{changepage}
\usepackage{amsmath}
\usepackage{amsthm}
\usepackage{amssymb}
\usepackage{xspace}
\usepackage{tikz}
\usepackage{amsmath,amsfonts,amssymb,amsthm}
\usepackage{mathtools}

%% Other
\graphicspath{{Figures/}}
\setlength\parindent{0pt}
\newcommand{\auth}{Giulia Baldini, Luis Fernandes, Agustin Vargas Toro}
\newcommand{\ass}{Assignment 5}

%% Page settings
\pagestyle{fancy}
\fancyhf{}
\rhead{\auth}
\lhead{\ass}
\rfoot{\thepage}
\newcommand{\real}{\mathbb{R}}
\newcommand{\inte}{\mathbb{Z}}
\newcommand{\liup}{\ensuremath{1}}
\newcommand{\lido}{\ensuremath{0}}
\newcommand{\liiup}{\ensuremath{\infty}}
\newcommand{\liido}{\ensuremath{1}}

\title{Foundations of Audio Signal Processing\\ \ass}
\author{\auth}

\begin{document}
	\maketitle
	\section*{Exercise 5.1}
	A signal $f$ is said to belong to the vector space $L^p(\real)$ if $\Vert f \Vert_p < \infty$.\\
	\textbf{a.} The $L^1(\real)$ norm of $f$:
	\begin{alignat*}{3}
	\Vert f \Vert_1 &= \int_{\lido}^{\liup} \vert f(t) \vert dt\\
	&= \int_{\lido}^{\liup} \frac{1}{\sqrt{t}} dt\\
	&= \left| 2\sqrt{t} \right|^{\liup}_{\lido}\\
	&= 2 - 0 &= 2
	\end{alignat*}
	So $f \in L^1(\real)$.\\
	The $L^2(\real)$ norm of $f$:
	\begin{alignat*}{3}
	\Vert f \Vert_2 &= \left(\int_{\lido}^{\liup} \vert f(t) \vert^2 dt\right)^\frac{1}{2}\\
	&= \left(\int_{\lido}^{\liup} \frac{1}{t} dt\right)^\frac{1}{2}\\
	&= \left(\left| \log(t) \right|^\liup_{\lido}\right)^\frac{1}{2}\\
	&= \left(0 + \infty \right)^\frac{1}{2} &= \infty
	\end{alignat*}
	So $f \notin L^2(\real)$.\\
	Thus $f \in L^1(\real) \backslash L^2(\real)$.\\
	
	\textbf{b.} The $L^1(\real)$ norm of $g$:
	\begin{alignat*}{3}
	\Vert g \Vert_1 &= \int_{\liido}^{\liiup} \vert g(t) \vert dt\\
	&= \int_{\liido}^{\liiup} \frac{1}{t} dt\\
	&= \left| \log(t) \right|^\liiup_{\liido} \\
	&= \infty  - 1 &= \infty
	\end{alignat*}
	So $f \notin L^1(\real)$.\\
	The $L^2(\real)$ norm of $g$:
	\begin{alignat*}{3}
	\Vert g \Vert_2 &= \left(\int_{\liido}^{\liiup} \vert g(t) \vert^2 dt\right)^\frac{1}{2}\\
	&= \left(\int_{\liido}^{\liiup} \frac{1}{t^2} dt\right)^\frac{1}{2}\\
	&= \left(\left| -\frac{1}{t} \right|^\infty_{\liido}\right)^\frac{1}{2}\\
	&= \left(-\frac{1}{\infty} + 1\right)^\frac{1}{2} &= 1
	\end{alignat*}
	So $f \in L^2(\real)$.\\
	Thus $f \in L^2(\real) \backslash L^1(\real)$.
	
	\section*{Exercise 5.2}
	\textbf{a.} $x(n) = e^n$\\
	According to Jensen's Inequality we have that
	\begin{equation*}
		\ell^1(\inte) \subset \ell^2(\inte) \subset \ell^3(\inte) \subset ... \subset \ell^\infty(\inte)
	\end{equation*}
	Since the $\ell^\infty$ norm is
	\begin{alignat*}{3}
	\Vert x(n) \Vert_\infty &= \sup_{n \in \inte}\vert x(n) \vert\\
	&=  \sup_{n \in \inte} e^n &= \infty
	\end{alignat*}
	and the series $\sum_{n\in\inte}^{\infty} x(n)$ diverges because of the geometric series convergence test, then we know that there is no $p \in [1, \infty]$ for which $x \in \ell^p(\inte)$.\\
	
	\textbf{b.} $x(n) = e^{2\pi i n}$\\
	Because of what we proved in (a), it holds that there is no $p \in [1, \infty]$ for which $x \in \ell^p(\inte)$. In fact, multiplying the value n by other values (even i) will not change the divergence of this series.\\
	
	\textbf{c.} $x(n) = \frac{1}{\sqrt{n}}, n > 0$\\
	We know that the $\Vert x(n) \Vert_2 = \frac{1}{n}$, which is divergent, so also  $\Vert x(n) \Vert_1 = \frac{1}{\sqrt{n}}$ is. Instead $\Vert x(n) \Vert_3 = \frac{1}{n^{\frac{3}{2}}}$ converges, because the numerator is a constant (and so does not increase) while the denominator is an increasing function, which means that it will eventually converge. Thus, according to Jensen's inequality, for $p \in [3, \infty]$ it holds that $x \in \ell^p(\inte)$.
	\section*{Exercise 5.3}
	\textbf{a-b.} The solutions can be found inside the \texttt{code} folder.
\end{document}
