\documentclass[12pt]{article}

%% Language and font encodings
\usepackage[english]{babel}
\usepackage[utf8x]{inputenc}
\usepackage[T1]{fontenc}
\usepackage{fancyhdr}

%% Sets page size and margins
\usepackage[a4paper,top=3cm,bottom=2cm,left=3cm,right=3cm,marginparwidth=1.75cm]{geometry}

%% Packages
\usepackage{xcolor}
\usepackage[colorlinks=true, allcolors=red]{hyperref}
\usepackage{lipsum}
\usepackage{graphicx}
\usepackage{float}
\usepackage[all]{hypcap}
\usepackage{changepage}
\usepackage{amsmath}
\usepackage{amsthm}
\usepackage{amssymb}
\usepackage{xspace}
\usepackage{tikz}
\usepackage{amsmath,amsfonts,amssymb,amsthm}
\usepackage{mathtools}

%% Other
\graphicspath{{Figures/}}
\setlength\parindent{0pt}
\newcommand{\auth}{Giulia Baldini, Luis Fernandes, Agustin Vargas Toro}
\newcommand{\ass}{Assignment 9}

%% Page settings
\pagestyle{fancy}
\fancyhf{}
\rhead{\auth}
\lhead{\ass}
\rfoot{\thepage}
\newcommand{\real}{\mathbb{R}}
\newcommand{\inte}{\mathbb{Z}}
\newcommand{\liup}{\ensuremath{1}}
\newcommand{\lido}{\ensuremath{0}}
\newcommand{\liiup}{\ensuremath{\infty}}
\newcommand{\liido}{\ensuremath{1}}
\newcommand{\eek}{\ensuremath{e^{-2\pi i \omega k}}}
\newcommand{\een}{\ensuremath{e^{-2\pi i \omega n}}}

\title{Foundations of Audio Signal Processing\\ \ass}
\author{\auth}

\begin{document}
	\maketitle
	\section*{Exercise 9.2}
	\subsection*{a} Upsampling operators are not time-invariant.\\
	Consider
	\begin{equation}
		\uparrow M[x](n) = \begin{cases}
		x(\frac{n}{M}), & \mbox{if } M\vert n\\ 
		0, & \mbox{otherwise}
		\end{cases}
	\end{equation}
	then
	\begin{equation}
		((\uparrow M)\circ T^k)[x](n) = \begin{cases}
		x(\frac{n}{M} - k), & \mbox{if } M\vert n\\
		0, & \mbox{otherwise}
		\end{cases}
	\end{equation}
	but
	\begin{equation}
	(T^k \circ (\uparrow M))[x](n) = \begin{cases}
	x(\frac{n-k}{M}), & \mbox{if } M\vert n\\
	0, & \mbox{otherwise}
	\end{cases}
	\end{equation}
	and since they are not equal, these operators are not time-invariant.
	
	\subsection*{b} Frequency-shift operators are time-invariant only in the $\omega = 0$ case.\\
	Consider
	\begin{equation}
		E_\omega[x](n) = e^{-2\pi i \omega n} \cdot x(n), \omega \in [0,1]
	\end{equation}
	then
	\begin{alignat}{2}
		(T^k \circ E_\omega)[x](n) &= e^{-2\pi i \omega n \cdot (- k)} \cdot x(n-k)\\
		&= e^{2\pi i \omega n k} \cdot x(n-k)
	\end{alignat}
	but
	\begin{alignat}{2}
		(E_\omega \circ T^k)[x](n) &= e^{-2\pi i \omega (n - k)} \cdot x(n-k)\\
		&= e^{-2\pi i \omega n + 2\pi i \omega k} \cdot x(n-k)
	\end{alignat}
	and since they are not equal (except when $\omega=0$, where then they both become $1 \cdot x(n-k)$), these operators are not time invariant.
\end{document}
