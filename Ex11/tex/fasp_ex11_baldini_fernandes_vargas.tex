\documentclass[12pt]{article}

%% Language and font encodings
\usepackage[english]{babel}
\usepackage[utf8x]{inputenc}
\usepackage[T1]{fontenc}
\usepackage{fancyhdr}

%% Sets page size and margins
\usepackage[a4paper,top=3cm,bottom=2cm,left=3cm,right=3cm,marginparwidth=1.75cm]{geometry}

%% Packages
\usepackage{xcolor}
\usepackage[colorlinks=true, allcolors=red]{hyperref}
\usepackage{lipsum}
\usepackage{graphicx}
\usepackage{float}
\usepackage[all]{hypcap}
\usepackage{changepage}
\usepackage{amsmath}
\usepackage{amsthm}
\usepackage{amssymb}
\usepackage{xspace}
\usepackage{tikz}
\usepackage{amsmath,amsfonts,amssymb,amsthm}
\usepackage{mathtools}

%% Other
\graphicspath{{Figures/}}
\setlength\parindent{0pt}
\newcommand{\auth}{Giulia Baldini, Luis Fernandes, Agustin Vargas Toro}
\newcommand{\ass}{Assignment 11}

%% Page settings
\pagestyle{fancy}
\fancyhf{}
\rhead{\auth}
\lhead{\ass}
\rfoot{\thepage}
\newcommand{\real}{\mathbb{R}}
\newcommand{\inte}{\mathbb{Z}}
\newcommand{\liup}{\ensuremath{1}}
\newcommand{\lido}{\ensuremath{0}}
\newcommand{\liiup}{\ensuremath{\infty}}
\newcommand{\liido}{\ensuremath{1}}
\newcommand{\eek}{\ensuremath{e^{-2\pi i \omega k}}}
\newcommand{\een}{\ensuremath{e^{-2\pi i \omega n}}}

\title{Foundations of Audio Signal Processing\\ \ass}
\author{\auth}

\begin{document}
	\maketitle
	\section*{Exercise 11.1}
	\textbf{a.}
	\section*{Exercise 11.2}
	\textbf{a.} Let us consider
	\begin{equation*}
	h(n) = 
	\begin{cases} 
	0.5 &\text{if } n \in \{0,1\},\\
	0 & \text{otherwise}.
	\end{cases}
	\end{equation*}
	\begin{equation*}
	g(n) = 
	\begin{cases} 
	0.5 &\text{if } n = 0,\\
	- 0.5 &\text{if } n = 1,\\
	0 & \text{otherwise}.
	\end{cases}
	\end{equation*}
	and 
	\begin{equation*}
		\sum_{N-1}^{n=0} q^n = \frac{1-q^N}{1-q}
	\end{equation*}
	and the fact that $N=2$ because we are considering a Haar lowpass-filter, then the frequency response $H^2$ can be written as:
	\begin{alignat*}{3}
		H^2 (\omega) &= \sum_{N-1}^{n=0} \frac{1}{N} \cdot \een\\
		&= \frac{1-e^{-2\pi i \omega N}}{1 - e^{-2\pi i \omega n}}\\
		&= \frac{1}{N} \cdot \frac{e^\frac{2\pi i \omega N}{2} - e^\frac{-2\pi i \omega N}{2}}{e^\frac{2\pi i \omega}{2} - e^\frac{-2\pi i \omega}{2}} \cdot \frac{e^\frac{-2\pi i \omega N}{2}}{e^\frac{-2\pi i \omega}{2}}\\
		&= \frac{1}{N} \cdot \frac{\sin(\pi \omega N)}{\sin(\pi \omega)} \cdot e^\frac{2\pi i \omega (N-1)}{2}\\
		&= \frac{1}{2} \cdot \frac{\sin(2\pi \omega)}{\sin(\pi \omega)} \cdot e^\frac{2\pi i \omega}{2}\\
		&= \frac{1}{2} \cdot \frac{0}{\sin(\pi \omega)} \cdot e^\frac{2\pi i \omega}{2} &= 0
	\end{alignat*}
\end{document}
