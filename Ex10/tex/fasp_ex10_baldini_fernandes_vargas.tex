\documentclass[12pt]{article}

%% Language and font encodings
\usepackage[english]{babel}
\usepackage[utf8x]{inputenc}
\usepackage[T1]{fontenc}
\usepackage{fancyhdr}

%% Sets page size and margins
\usepackage[a4paper,top=3cm,bottom=2cm,left=3cm,right=3cm,marginparwidth=1.75cm]{geometry}

%% Packages
\usepackage{xcolor}
\usepackage[colorlinks=true, allcolors=red]{hyperref}
\usepackage{lipsum}
\usepackage{graphicx}
\usepackage{float}
\usepackage[all]{hypcap}
\usepackage{changepage}
\usepackage{amsmath}
\usepackage{amsthm}
\usepackage{amssymb}
\usepackage{xspace}
\usepackage{tikz}
\usepackage{amsmath,amsfonts,amssymb,amsthm}
\usepackage{mathtools}

%% Other
\graphicspath{{Figures/}}
\setlength\parindent{0pt}
\newcommand{\auth}{Giulia Baldini, Luis Fernandes, Agustin Vargas Toro}
\newcommand{\ass}{Assignment 10}

%% Page settings
\pagestyle{fancy}
\fancyhf{}
\rhead{\auth}
\lhead{\ass}
\rfoot{\thepage}
\newcommand{\real}{\mathbb{R}}
\newcommand{\inte}{\mathbb{Z}}
\newcommand{\liup}{\ensuremath{1}}
\newcommand{\lido}{\ensuremath{0}}
\newcommand{\liiup}{\ensuremath{\infty}}
\newcommand{\liido}{\ensuremath{1}}
\newcommand{\eek}{\ensuremath{e^{-2\pi i \omega k}}}
\newcommand{\een}{\ensuremath{e^{-2\pi i \omega n}}}

\title{Foundations of Audio Signal Processing\\ \ass}
\author{\auth}

\begin{document}
	\maketitle
	\section*{Exercise 10.1}
	    \textbf{b-c.}
	    
	        Let $ x(n) = (x(0), x(1), x(2)) = (1, 2, 3)$, $ y(n) = (y(0), y(1)) = (4, 5)$.
	        Let $ z(n) = (x * y)(n)$ \newline
	        
	        $$z(0) = \sum_{k \in \mathbb{Z}} x(k).y(0 - k) = x(0).y(0) = 1.4 = 4$$
	        $$z(1) = \sum_{k \in \mathbb{Z}} x(k).y(1 - k) = x(0).y(1) + x(1).y(0) = 1.5 + 2.4 = 13$$
	        $$z(2) = \sum_{k \in \mathbb{Z}} x(k).y(2 - k) = x(1).y(1) + x(2).y(0) = 2.5 + 3.4 = 22$$
	        $$z(3) = \sum_{k \in \mathbb{Z}} x(k).y(3 - k) = x(2).y(1) = 3.5 = 15$$
	        
	        Therefore, $z(n) = (x*y)(n) = (z(0), z(1), z(2), z(3)) = (4, 13, 22, 15)$ \newline
	        
	        
	        
	        
	    \textbf{b-c.} The solutions can be found inside the \texttt{code} folder.
	\section*{Exercise 10.2}
	\section*{Exercise 10.3}
	\textbf{a-c.} The solutions can be found inside the \texttt{code} folder.
\end{document}
